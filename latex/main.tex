\documentclass[10pt]{article}
\usepackage[a4paper, left=1.5cm, right=1.5cm, top=3.5cm]{geometry}
\usepackage[ngerman]{babel}
\usepackage{graphicx}
%\usepackage{multicol}
%\usepackage{amssymb}
%\usepackage{titlesec}
%\usepackage{wrapfig}
%\usepackage{blindtext}
%\usepackage{lipsum}
%\usepackage{caption}
%\usepackage{listings}
%\usepackage{fancyhdr}
%\usepackage{nopageno}
%\usepackage{authblk}
%\usepackage{amsmath} % tons of math stuff
%\usepackage{mathtools} % e.g. alignment within matrix
%\usepackage{bm} % provides shorthand for bold in math mode
%\usepackage[ISO]{diffcoeff}
%\usepackage{xcolor}
%\usepackage{csquotes} % e.g. provides \enquote
%\fancyhf[]{}
\usepackage{booktabs}
\usepackage{atbegshi}% http://ctan.org/pkg/atbegshi

%\AtBeginDocument{\AtBeginShipoutNext{\AtBeginShipoutDiscard}} % remove empty page
%\addtocounter{page}{-1}

\begin{document}


\begin{Figure}
    \includegraphics[width=1.0\textwidth, angle=0]{stability.pdf}
\caption{
    Stabilität in Abhängigkeit der Unteren Grenze des Fit-Intervalls
    und Vergleich mit der effektiven Masse.
}
\label{fig:stability}
\end{Figure}

\begin{Figure}
    \includegraphics[width=1.0\textwidth, angle=0]{correlator.pdf}
\caption{
    Pion-Pion-Korrelator (gemittelt über die verschiedenen Konfigurationen) in Abhängigkeit der Zeit,
    mit Fit zur Bestimmung der Pion-Masse.
}
\label{fig:correlation}
\end{Figure}

\begin{center}
    \include{example_results}
    \caption{
        Fit-Parameter. $\delta X$: Fehler, berechnet mit Bootstrapping. $\overline X_B$: Bootstrap-Mittelwert.
    }
    \label{tab:results}
\end{center}

\end{document}
